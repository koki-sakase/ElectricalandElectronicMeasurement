\documentclass[a4paper,11pt]{jsarticle}


% 数式
\usepackage{amsmath,amsfonts}
\usepackage{bm}
\usepackage{siunitx}
% 画像
\usepackage[dvipdfmx]{graphicx}
\usepackage{booktabs}


\begin{document}

\title{}
\author{}
\date{\today}


\section{$0\mathrm{\,\si{\celsius}}$における抵抗、抵抗温度係数が、それぞれ$R_{1},\, \alpha _1$と$R_{2},\, \alpha _{2}$である2つの抵抗線を直列に接続する。合成抵抗温度係数を求めなさい。}

全体の合成抵抗を$R$、温度を$t$、求める合成抵抗温度計数を$\alpha $とすると、
\begin{align}
  \begin{split}
    R(1+\alpha t)&=R_{1}(1+\alpha _{1}t)+R_{2}(1+\alpha _{2}t)\\
    R+\alpha tR&=R_{1}+\alpha _{1}tR_{1}+R_{2}+\alpha _{2}tR_{2}\\
    &=R_{1}+R_{2}+t(\alpha _{1}R_{1}+\alpha _{2}R_{2})\label{eq1}
  \end{split}
\end{align}
よって、両辺を比較すると、
\begin{align}
  R&=R_{1}+R{2}\label{eq2}\\
  \alpha R&=\alpha _{1}R_{1}+\alpha _{2}R_{2}\label{eq3}
\end{align}
したがって、式\eqref{eq2},~\eqref{eq3}より、
\begin{align}
  \begin{split}
    \alpha (R_{1}+R{2})
    &=\alpha _{1}R_{1}+\alpha _{2}R_{2}\\
    \alpha
    &=\frac{\alpha _{1}R_{1}+\alpha _{2}R_{2}}{R_{1}+R{2}}
    \label{eq4}
  \end{split}
\end{align}
となる。

\end{document}