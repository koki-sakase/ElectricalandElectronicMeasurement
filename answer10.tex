\documentclass[a4paper,11pt]{jsarticle}


% 数式
\usepackage{amsmath,amsfonts}
\usepackage{bm}
\usepackage{siunitx}
% 画像
\usepackage[dvipdfmx]{graphicx}
\usepackage{booktabs}


\begin{document}

\title{}
\author{}
\date{\today}


\section{「定格電圧$15\mathrm{\,\si{\volt}}$、$1.5$級の電圧計」と「定格電圧$150\mathrm{\, \si{\volt}}$、$0.5$級の電圧計」で$10\mathrm{\,\si{\volt}}$の電圧を測定した。誤差が小さいのはどちらですか、理由を添えて答えなさい。}

講義資料$1(\mathrm{p}. 52)$より、指示計器の正確さによる分類は、表\ref{tab1}となる。
\begin{table}[htbp]
  \centering
  \caption{指示計器の正確さによる分類}\label{tab1}
  \begin{tabular}{cll}
    \toprule
    階級&許容誤差&主な用途\\
    \midrule
    0.2級&$\pm 0.2 \%$&標準機~(副)\\
    0.5級&$\pm 0.5 \%$&精密測定\\
    1.0級&$\pm 1.0 \%$&普通の測定\\
    1.5級&$\pm 1.5 \%$&工業用の普通の測定\\
    2.5級&$\pm 2.5 \%$&正確さに重きを置かない測定\\
    \bottomrule
  \end{tabular}
\end{table}

したがって、「定格電圧$15\mathrm{\,\si{\volt}}$、$1.5$級の電圧計」の最大許容誤差$\delta _{1}$は、
\begin{align}
  \begin{split}
    \delta _{1}=15\times 0.015=0.225 \mathrm{\,\si{\volt}}
  \end{split}
\end{align}
となり、「定格電圧$150\mathrm{\, \si{\volt}}$、$0.5$級の電圧計」の最大許容誤差$\delta _{2}$は、
\begin{align}
  \begin{split}
    \delta _{2}=150 \times 0.005 = 0.65 \mathrm{\,\si{\volt}}
  \end{split}
\end{align}
となる。したがって、$\delta _{1}<\delta _{2}$となるため、誤差が小さいのは「定格電圧$15\mathrm{\,\si{\volt}}$、$1.5$級の電圧計」となる。
\end{document}