\documentclass[a4paper,11pt]{jsarticle}


% 数式
\usepackage{amsmath,amsfonts}
\usepackage{bm}
% 画像
\usepackage[dvipdfmx]{graphicx}


\begin{document}

\title{a}
\author{a}
\date{\today}

\section*{問題$9$. 抵抗値$R$、長さ$l$、直径$d$を測定して、円柱状導体の抵抗率$\rho $を間接測定する。誤差伝搬の法則を導出しなさい。}

抵抗$R$は、長さ$l$、直径$d$、抵抗率$\rho $の円柱状導体のとき
\begin{align}
  \begin{split}
    R=\rho \times \frac{l}{\pi {\left(\frac{d}{2}\right)}^{2}}
    \label{eq1}
  \end{split}
\end{align}
したがって、
\begin{align}
  \begin{split}
    \rho
    &=R \times \frac{{\pi \left(\frac{d}{2}\right)}^{2}}{l}\\
    &=\frac{\pi R d^{2}}{4l}\\
    &=\frac{\pi }{4}R^{1} d^{2} l^{-1}
    \label{eq2}
  \end{split}
\end{align}

講義資料$1(\mathrm{p}. 26)$より、誤差率の一般的な関数は、
\begin{align}
  \begin{split}
    \left|\Delta y\right|\leq \left|\frac{\partial f}{\partial x_{1}}\Delta x_{1}\right|+\left|\frac{\partial f}{\partial x_{2}}\Delta x_{2}\right|+\left|\frac{\partial f}{\partial x_{3}}\Delta x_{3}\right|+\cdots +\left|\frac{\partial f}{\partial x_{n}}\Delta x_{n}\right|
    \label{eq3}
  \end{split}
\end{align}
となる。したがって、式\eqref{eq2}$,\ $\eqref{eq3}より、
\begin{align}
  \begin{split}
    \left|\Delta \rho\right|
    &\leq \left|\frac{\partial \rho }{\partial R}\Delta R\right|+\left|\frac{\partial \rho }{\partial d}\Delta d\right|+\left|\frac{\partial \rho }{\partial l}\Delta l\right|\\
    &\leq \left|\frac{\pi d^{2}}{4l}\Delta R\right|+\left|\frac{2\pi R d}{4l}\Delta d\right|+\left|\frac{-\pi R d^{2}}{4l^{2}}\Delta l\right|\\
    \left|\frac{\Delta \rho}{\rho}\right|
    &\leq \left|\frac{\frac{\pi d^{2}}{4l}}{\frac{\pi R d^{2}}{4l}}\Delta R\right|+\left|\frac{\frac{2\pi R d}{4l}}{\frac{\pi R d^{2}}{4l}}\Delta d\right|+\left|\frac{\frac{-\pi R d^{2}}{4l^{2}}}{\frac{\pi R d^{2}}{4l}}\Delta l\right|\\
    &\leq \left|\frac{\Delta R}{R}\right|+2\left|\frac{\Delta d}{d}\right|-\left|\frac{\Delta l}{l}\right|
    \label{eq4}
  \end{split}
\end{align}

となり、題意における誤差伝搬の法則は式\eqref{eq4}となる。

\end{document}